\documentclass[12pt,a4paper]{article}
\usepackage[utf8]{inputenc}
\usepackage{amsmath}
\usepackage{amsfonts}
\usepackage{amssymb}
\usepackage[ngerman]{babel}
\parindent0pt
\author{Matthias Englert, Fabian Schilha, Andreas Rottach}
\title{Pflichtenheft}
\begin{document}
\maketitle
\newpage
\tableofcontents
\newpage
\section{Überblick}
\subsection{Einleitung}
Dieses Software-Projekt hat sich als Ziel gesetzt eine webbasierte, zentrale E-Learning Plattform für die Studenten der Universität Ulm bereitzustellen. Das System soll die Lerninhalte individuell für jeden Benutzer in geeigneter Form strukturieren. Des Weiteren soll jeder Anwender den Lernstoff erweitern und mit anderen darüber diskutieren können. Die Lerninhalte sollen in einer hierarchischen Struktur mit verschiedenen Detailebenen dargestellt werden, um unterschiedliche Einblicke in ein Themengebiet zu ermöglichen. Das Skript soll durch verschiedene digitale Inhalte wie Bilder, Texte oder Videos unterstützt werden. Dozenten können initiale Lehrinhalte bereitstellen, die sich im Laufe des Semesters verändern oder erweitern werden können. 
Jeder Student soll die Möglichkeit haben, mit andern über Probleme zu diskutieren und Lösungen zu finden.

\subsection{Motivation}
Zurzeit verfügt die Universität Ulm über viele Plattformen (Moodle, ILIAS, Rubikon und slc) um Vorlesungsmaterialien den Studenten bereitzustellen. Diese Plattformen sind keine echten E-Learning Systeme, da man sie nur nutzt um Dokumente wie Skripte oder Übungsblätter herunterzuladen. Außerdem gibt es als einzige Informationsquelle zum Lernen nur das Skript und keine anderen Medien wie z.B. Videos. Das Skript kann dabei nur in einer festen linearen Struktur durchgearbeitet werden. Lernen ist allerdings kein linearer Prozess, sondern ein Prozess, bei dem Informationen zu einem Netzwerk zusammengebaut werden. Dieses Netzwerk zu erweitern und immer wieder umzustrukturieren stellt den eigentlichen Lernprozess da. Bei einem linearen Skript fehlen dabei Querverweise zu anderen Quellen, falls man einen Begriff beispielsweise nicht versteht. Zu diesem Lernprozess gehört auch, dass man sich mit anderen Studenten austauscht. In den bereits vorhandenen Vorlesungsplattformen lädt jedoch jeder das Skript runter und lernt für sich allein. Es gibt keine Möglichkeit persönliche Notizen im Skript mit anderen zu teilen. Dadurch bekommt auch der Dozent keine Vorstellung davon was man im Skript besser machen könnte, sodass sich das Skript über die Jahre kaum ändert.
Mit unserem E-Learning System wollen wir diese Probleme anpacken! 

\subsection{Vision und Leitbild}
Das Ziel des Projekts ist es den Studenten für jede Vorlesung eine zentrale webbasierte Lernumgebung anzubieten. Der Dozent einer Vorlesung hat die Möglichkeit eine Veranstaltung anzulegen, auf der er dann ein initiales Skript bereitstellen kann. Durch die während des Semesters aufkommenden Diskussionen ist er in der Lage das Skript mit Hilfe der Studenten zu erweitern. Die Vorlesungsinhalte sollen dabei nicht mehr linear aufgebaut sein, sondern einzelne Teile (z.B. eine Definition oder ein Satz in der Mathematik) sollen auf Karteikarten gespeichert werden. Die Karteikarten sind hierarchisch angeordnet und zusätzlich durch Querverweise miteinander verknüpft werden, sodass ein Netzwerk entsteht. Dadurch ist es für die Anzeige beispielsweise möglich auf Vorlesungsfolien weniger Information zu packen, als ins Skript, sodass die Anzeige flexibel wird. Durch die Struktur als Netzwerk ist es für einen Student, der beispielsweise ein Matheskript liest und über den Begriff der Differenzialgleichung stößt, möglich zuerst eine kurze Definition zu dem Begriff zu erhalten. Falls dies nicht ausreichend ist, hat er die Wahl sich zwischen verschiedenen Quellen zu diesem Thema zu entscheiden. Beispielsweise könnte er auf ein YouTube-Video oder eine andere Website verlinkt werden. In dem Netzwerk ist es aber trotzdem noch wichtig dass es einen linearen Pfad gibt, der das Skript repräsentiert. Des weiteren soll ein Student zu jeder Karteikarte Notizen machen oder eine Diskussion anstoßen können. Der Student kann entscheiden, ob andere seine Notizen sehen dürfen. Um die Qualität der Diskussion beurteilen zu können, gibt es die Möglichkeit, einzelne Beiträge durch positive Bewertungen hervorzuheben. Außerdem existieren Moderatoren, die die Aufgabe haben, schlechte Beiträge zu entfernen und besonders gute Beiträge ins Skript einzuarbeiten. Die Rolle des Moderators kann z.B. der Dozent oder der Übungsleiter übernehmen.



\section{Projektkontext}
Das Software-System wird im Rahmen des Softwaregrundprojekts Wintersemester 2014/2015 im Bereich Informatik entstehen. Dies kann eventuell in den Bestehenden Lehrbetrieb der Universität Ulm eingebettet werden, so dass allen Studenten an der Universität die Möglichkeit zu diesem System angeboten werden kann.

\section{Anforderungsanalyse}
\subsection{Fachwissen (Glossar)}
\begin{tabular}{l l} 
BEGRIFF & Student \\ 
BESCHREIBUNG & Immatrikulierte Person an einer Universität \\ 
ISTEIN & Benutzer \\
KANNSEIN & Anwender, Administrator \\ 
ASPEKT & erweitert die Inhalte des Systems und stellt diese anderen Benutzern zur Verfügung \\
BEISPIEL & Fabian Schilha\\

&\\ 

BEGRIFF & Administrator \\ 
BESCHREIBUNG & verwaltet die Benutzer des Systems und administriert die Zugänge zu dem System \\ 
ISTEIN & Benutzer \\
KANNSEIN & Anwender, Administrator \\ 
ASPEKT & erweitert die Inhalte des Systems und stellt diese anderen Benutzern zur Verfügung \\
BEISPIEL & Fabian Schilha\\
\end{tabular}


\subsection{Systemkontext}
\subsubsection{Akteure und Anwendungsfälle}
In diesem Abschnitt werden die beteiligten Akteure identifiziert. Danach werden alle auftretenden Anwendungsfälle durch Anwendungsfalldiagrame dargestellt.
Folgende Akteue sind am System beteiligt.\\\\

\begin{tabular}{l p{10cm}}
\textbf{Akteuer} & Benutzer \\ 
\hline \textbf{Beschreibung} & Ein Benutzer kann sich am System anmelden und für Kurse registrieren. \\ 
\hline 
\end{tabular}\\\\

\begin{tabular}{l p{10cm}}
\textbf{Akteuer} & Dozent \\ 
\hline \textbf{Beschreibung} & Ein Dozent leitet eine Veranstaltung und überwacht diese. Er erstellt ein initiales Skript und steuert, wie sich dieses weiterentwickelt. Außerdem kann er Kommentare zu Diskussionen hinterlassen. Er hat die vollständige Kontrolle über eine Veranstaltung.\\ 
\hline 
\end{tabular}\\\\

\begin{tabular}{l p{10cm}}
\textbf{Akteuer} & Student \\ 
\hline \textbf{Beschreibung} & Ein Student kann sich zu Veranstaltungen anmelden. Er kann an Diskussionen teilnehmen und den Lernstoff erweitern.\\ 
\hline 
\end{tabular}\\\\

\begin{tabular}{l p{10cm}}
\textbf{Akteuer} & Moderator \\ 
\hline \textbf{Beschreibung} & Ein Moderator überwacht Diskussionen. Er überträgt gute Kommentare in den Lernstoff und verbirgt nutzlose Aussagen.\\ 
\hline 
\end{tabular}\\\\

\begin{tabular}{l p{10cm}}
\textbf{Akteuer} & Administrator \\ 
\hline \textbf{Beschreibung} & Ein Administrator hat vollständigen Zugriff auf alle Veranstaltungen und ist dafür verantwortlich, auftretende Probleme zu lösen. \\ 
\hline 
\end{tabular}\\\\

\begin{tabular}{l p{10cm}}
\textbf{Akteuer} & eMail-Server \\ 
\hline \textbf{Beschreibung} & Ein eMail-Server ist für die externe Kommunikation mit den Nutzern zuständig. Er versendet Bestätigungs-Mails oder weißt auf bestimmte Änderungen hin. \\ 
\hline 
\end{tabular}\\\\

\subsubsection{Szenarien}
\subsubsection{Systemaufgabe}
Hier werden alle Systemaufgaben, die dazugehörigen Teilnehmer und jeweils eine kurze Beschreibung aufgelistet. Außerdem wird jede Anforderung mit einer Markierung(-2 bis 2) versehen, die darlegt, wie wichtig diese Anforderung ist.
\paragraph{Funktionale Anforderungen}\mbox{}\\

\begin{tabular}{l p{10cm}}
\multicolumn{2}{l}{\textbf{Anmelden}} \\ \hline
\textbf{Bedeutung} & 2 \\ \hline 
\textbf{Beteiligt} & Benutzer \\ \hline 
\textbf{Beschreibung} & Der Benutzer gibt seine Anmeldeinformationen ein. Falls diese richtig sind, wird eine Session eröffnet und er ist eingeloggt. Andernfalls erhält er eine Fehlermeldung. \\ 
\hline 
\end{tabular}\\\\

\begin{tabular}{l p{10cm}}
\multicolumn{2}{l}{\textbf{Abmelden}} \\ \hline
\textbf{Bedeutung} & 2 \\ \hline 
\textbf{Beteiligt} & Benutzer \\ \hline 
\textbf{Beschreibung} & Der Benutzer wird von System abgemeldet. Die Session wird beendet und er landet wieder auf der Startseite. \\ 
\hline 
\end{tabular}\\\\

\begin{tabular}{l p{10cm}}
\multicolumn{2}{l}{\textbf{Zu Veranstaltung anmelden}} \\ \hline
\textbf{Bedeutung} & 2 \\ \hline 
\textbf{Beteiligt} & Benutzer, eMail-Server \\ \hline 
\textbf{Beschreibung} & Der Benutzer begibt sich auch die Veranstaltungsseite und will sich zu dieser Veranstaltung anmelden. Dazu muss er optional einen Zugangscode eingeben. Falls dieser Code korrekt ist, wird die Zugangsberechtigung vom System gespeichert und die Anmeldung war erfolgreich. Andernfalls wird der Benutzer über das Fehlschlagen der Anmeldung informiert. \\ 
\hline 
\end{tabular}\\\\

\begin{tabular}{l p{10cm}}
\multicolumn{2}{l}{\textbf{Veranstaltung anzeigen}} \\ \hline
\textbf{Bedeutung} & 1 \\ \hline 
\textbf{Beteiligt} & Benutzer \\ \hline 
\textbf{Beschreibung} & Ein Benutzer wählt eine Veranstaltungsseite aus. Das System prüft ob dieser Nutzer Zugang zu dieser Seite hat und zeigt die Seite gegebenenfalls an. Sonst wird nur eine kleine Veranstaltungsinformation angezeigt. \\ 
\hline 
\end{tabular}\\\\

\begin{tabular}{l p{10cm}}
\multicolumn{2}{l}{\textbf{Von Veranstaltung abmelden}} \\ \hline
\textbf{Bedeutung} & 0 \\ \hline 
\textbf{Beteiligt} & Benutzer, eMail-Server \\ \hline 
\textbf{Beschreibung} & Der Benutzer kann seine Teilnahme an einer Veranstaltung beenden. \\ 
\hline 
\end{tabular}\\\\

\begin{tabular}{l p{10cm}}
\multicolumn{2}{l}{\textbf{Fehlermeldung anzeigen}} \\ \hline
\textbf{Bedeutung} & 1 \\ \hline 
\textbf{Beteiligt} & Benutzer \\ \hline 
\textbf{Beschreibung} & Wenn ein Fehler auftritt wird er dem Benutzer angezeigt. Außerdem wird eine Lösung des Problems dargelegt, falls dies möglich ist. \\ 
\hline 
\end{tabular}\\\\

\begin{tabular}{l p{10cm}}
\multicolumn{2}{l}{\textbf{Notizen machen}} \\ \hline
\textbf{Bedeutung} & -1 \\ \hline 
\textbf{Beteiligt} & Benutzer \\ \hline 
\textbf{Beschreibung} & Ein Benutzer kann sich zu einer bestimmten Karteikarte Notizen machen. Das System verknüpft diese Notiz dann mit der entsprechenden Karteikarte. \\ 
\hline 
\end{tabular}\\\\

\begin{tabular}{l p{10cm}}
\multicolumn{2}{l}{\textbf{Initiales Skript importieren}} \\ \hline
\textbf{Bedeutung} & -2 \\ \hline 
\textbf{Beteiligt} & Dozent \\ \hline 
\textbf{Beschreibung} & Der Dozent hat die Möglichkeit, ein initiales Skript zu importieren. Das System konvertiert dieses Skript in eine Sammlung von Karteikarten. \\ 
\hline 
\end{tabular}\\\\

\begin{tabular}{l p{10cm}}
\multicolumn{2}{l}{\textbf{Skript exportieren}} \\ \hline
\textbf{Bedeutung} & -1 \\ \hline 
\textbf{Beteiligt} & Benutzer \\ \hline 
\textbf{Beschreibung} & Das System wandelt die Karteikarten-Repräsentation der Lerninhalte in ein vom Benutzer gewähltes Format (z.b. PDF) mit einem bestimmten Detail-Grad um und bietet es ihm zum Download an. \\ 
\hline 
\end{tabular}\\\\

\begin{tabular}{l p{10cm}}
\multicolumn{2}{l}{\textbf{Diskussion erstellen}} \\ \hline
\textbf{Bedeutung} & 1 \\ \hline 
\textbf{Beteiligt} & Benutzer \\ \hline 
\textbf{Beschreibung} & Der Benutzer möchte eine Diskussion zu einer bestimmten Karteikarte starten. Das System legt diese Diskussion an und bietet dem Benutzer die Möglichkeit Kommentare zu verfassen. \\ 
\hline 
\end{tabular}\\\\

\begin{tabular}{l p{10cm}}
\multicolumn{2}{l}{\textbf{Lerninhalte einsehen}} \\ \hline
\textbf{Bedeutung} & 2 \\ \hline 
\textbf{Beteiligt} & Benutzer \\ \hline 
\textbf{Beschreibung} & Der Benutzer wählt den Lerninhalt einer Veranstaltung aus. Das System fragt ihn dann nach der Darstellungsform und zeigt die Inhalte dementsprechend an. \\ 
\hline 
\end{tabular}\\\\

\begin{tabular}{l p{10cm}}
\multicolumn{2}{l}{\textbf{Veranstaltung anlegen}} \\ \hline
\textbf{Bedeutung} & 2 \\ \hline 
\textbf{Beteiligt} & Dozent \\ \hline 
\textbf{Beschreibung} & Ein Dozent kann eine neue Veranstaltung anlegen. Dazu gibt er die Veranstaltungsdaten ein. Daraus erzeugt das System dann eine neue Veranstaltung.\\ 
\hline 
\end{tabular}\\\\

\begin{tabular}{l p{10cm}}
\multicolumn{2}{l}{\textbf{Moderator ernennen}} \\ \hline
\textbf{Bedeutung} & 1 \\ \hline 
\textbf{Beteiligt} & Dozent, Moderator \\ \hline 
\textbf{Beschreibung} & Ein Dozent kann einen Moderator ernennen, indem er die Veranstaltungseinstellungen ändert.\\ 
\hline 
\end{tabular}\\\\

\begin{tabular}{l p{10cm}}
\multicolumn{2}{l}{\textbf{Veranstaltung bearbeiten}} \\ \hline
\textbf{Bedeutung} & 1 \\ \hline 
\textbf{Beteiligt} & Dozent \\ \hline 
\textbf{Beschreibung} & Ein Dozent kann eine bestehende Veranstaltung editieren. Er kann auch einstellen, ob das Skript editierbar sein soll oder nicht, usw..\\ 
\hline 
\end{tabular}\\\\

\begin{tabular}{l p{10cm}}
\multicolumn{2}{l}{\textbf{Karteikarte ändern}} \\ \hline
\textbf{Bedeutung} & 2 \\ \hline 
\textbf{Beteiligt} & Dozent, Moderator \\ \hline 
\textbf{Beschreibung} & Ein Dozent oder Moderator kann eine bestehene Karteikarte ändern. Er kann Name, Inhalt oder Verlinkungen editieren.\\ 
\hline 
\end{tabular}\\\\

\begin{tabular}{l p{10cm}}
\multicolumn{2}{l}{\textbf{Karteikarte hinzufügen}} \\ \hline
\textbf{Bedeutung} & 2 \\ \hline 
\textbf{Beteiligt} & Dozent, Moderator \\ \hline 
\textbf{Beschreibung} & Ein Dozent oder Moderator kann eine neue Karteikarte hinzufügen.\\ 
\hline 
\end{tabular}\\\\

\begin{tabular}{l p{10cm}}
\multicolumn{2}{l}{\textbf{Kommentare entfernen}} \\ \hline
\textbf{Bedeutung} & -1 \\ \hline 
\textbf{Beteiligt} & Dozent, Moderator \\ \hline 
\textbf{Beschreibung} & Ein Dozent oder Moderator kann einen bestehenden Kommentar entfernen.\\ 
\hline 
\end{tabular}\\\\

\begin{tabular}{l p{10cm}}
\multicolumn{2}{l}{\textbf{Kommentare bewerten}} \\ \hline
\textbf{Bedeutung} & 0 \\ \hline 
\textbf{Beteiligt} & Dozent, Moderator, Benutzer \\ \hline 
\textbf{Beschreibung} & Alle Teilnehmer einer Veranstaltung können Kommentare mit positiven Bewertungen versehen. Diese werden dann hervorgehoben.\\ 
\hline 
\end{tabular}\\\\

\begin{tabular}{l p{10cm}}
\multicolumn{2}{l}{\textbf{Veranstaltung löschen}} \\ \hline
\textbf{Bedeutung} & -1 \\ \hline 
\textbf{Beteiligt} & Dozent \\ \hline 
\textbf{Beschreibung} & Der Dozent kann seine selbst erstellten Veranstaltungen löschen.\\ 
\hline 
\end{tabular}\\\\

\paragraph{Nicht funktionale Anforderungen}\mbox{}\\
Hier werden alle nicht funktionalen Anforderungen aufgelistet, denen das System gerecht werden muss. Auch hier wird jeder Abschnitt mit einer Nummer zwischen -2 und 2 versehen. Diese Nummer repräsentiert auch hier, wie wichtig diese Anforderung für das System ist.
\subparagraph{Benutzerfreundlichkeit (2)}
\begin{itemize}
\item Ein noch so gut funktionierendes System ist wertlos, wenn die Handhabung des Systems so schlecht ist, dass sich kein Anwender lange damit auseinandersetzen will. 
\item Es muss intuitiv und einfach zu bedienen sein.
\end{itemize}
\subparagraph{Robustheit (1)}
\begin{itemize}
\item Das System muss robust gegenüber Abstürzen sein. 
\item Es sollten keine unerwarteten Zustände auftreffen. Und falls doch, sollte sich das System so verhalten, dass keine Daten verlohren gehen.
\end{itemize}
\subparagraph{Performance (0)}
\begin{itemize}
\item Das System sollte effizient sein.
\item Viele Datenbankzugriffe erfordern eine effiziente Strukturierung der Daten.
\item Es sollte auf langsame Web-Plugins verzichtet werden. Diese beeinträchtigen nur die Geschwindigkeit des System auf den unterschiedlichen Browsern.
\end{itemize}
\subparagraph{Sicherheit (1)}
\begin{itemize}
\item Die Rechteregelung sollte einwandfrei funktionieren.
\item Die privaten Daten wie z.b. Notizen sollten nur vom Erzeuger eingesehn werden können.
\item Verbindungen sollten immer verschlüsselt sein.
\end{itemize}
\subparagraph{Verfügbarkeit (1)}
\begin{itemize}
\item Das System sollte nicht nur aus dem Uni-Netz sondern auch Weltweit über das Web genutzt werden können.
\item Es sollte zu Wartungszwecken nicht abgeschaltet werden müssen.
\end{itemize}
\subparagraph{Wartbarkeit (-1)}
\begin{itemize}
\item Es sollte eine eigene Schnittstelle für Administratoren geben. Dies erleichtert die Wartung des Systems enorm.
\end{itemize}

\end{document}
